% Document setup
\documentclass[11pt]{book}

% Location of the csas-style repository: adjust path as needed
\newcommand{\locRepo}{csas-style}

% Use the style file in the csas-style repository (sr.sty)
\usepackage{\locRepo/sr-french}

% header-includes from R markdown entry
$header-includes$

%%%% Commands for title page etc %%%%%

% Title
\newcommand{\rdTitle}{$title$}

% English title
\newcommand{\rdTitleFr}{$title_other$}

% Make the PDF bookmark title uppercase
\ExplSyntaxOn
\cs_set_eq:NN\textupper\text_uppercase:n
\ExplSyntaxOff
% \newcommand\textupper{}
% Last line needed according to
% https://tex.stackexchange.com/questions/548437/how-can-i-use-expl3-text-uppercase-for-hyperref-bookmarks-with-german-umlauts

\newcommand{\rdTitleFrUpper}{\textupper{\rdTitleFr{}}}

% Title short
\newcommand{\rdTitleShort}{$title_short$}
% End of title short

% Publication year
\newcommand{\rdYear}{$year$}

% Publication month
\newcommand{\rdMonth}{$month$}

% Report number
\newcommand{\rdNumber}{\threedigits{$report_number$}}
% End of report number

% Approver (name\\position)
\newcommand{\rdApp}{$approver$}

% Approval date
\newcommand{\rdAppDate}{$approval_day$ $approval_month$ $approval_year$}

% SR/Work done date
\newcommand{\rdWorkDoneMonth}{$work_done_month$}
\newcommand{\rdWorkDoneYear}{$work_done_year$}

% Branch
\newcommand{\rdBranch}{$branch$}

% Region
\newcommand{\rdRegion}{$region$}

\newcommand{\rdISBN}{$isbn$}
\newcommand{\rdCatNo}{$cat_no$}

%%%% End of title page commands %%%%%

% commands and environments needed by pandoc snippets
% extracted from the output of `pandoc -s`
%% Make R markdown code chunks work
\usepackage{array}
\usepackage{amssymb,amsmath}
\usepackage{color}
\usepackage{fancyvrb}

% for \textdegree in No Cat...
\usepackage{siunitx}
\usepackage{textcomp}

% From default template:
\newcommand{\VerbBar}{|}
\newcommand{\VERB}{\Verb[commandchars=\\\{\}]}
% Add theme here

\newcommand{\lt}{\ensuremath <}
\newcommand{\gt}{\ensuremath >}


%Defines cslreferences environment
%Required by pandoc 2.8
%Copied from https://github.com/rstudio/rmarkdown/issues/1649
$if(csl-refs)$
  \newlength{\cslhangindent}
  \setlength{\cslhangindent}{1.5em}
  \newenvironment{cslreferences}{
  $if(csl-hanging-indent)$
    \everypar{\setlength{\hangindent}{\cslhangindent}}
  $endif$}%
 {\par}
$endif$

\DeclareGraphicsExtensions{.png,.pdf}

\begin{document}

\bookmark[dest=PageOne]{\rdTitleFrUpper{}}

\renewcommand{\tablename}{Tableau}

\MakeFirstPage

$body$

\clearpage

\chapter{Le pr\'{e}sent rapport est disponible aupr\`{e}s du :}

\begin{center}
Centre des avis scientifiques (CAS)\\
\rdRegion{}\\
P\^{e}ches et Oc\'{e}ans Canada\\
AddressPlaceholder\\
\vspace{0.2cm}
Courriel : EmailPlaceholder\\
Adresse internet : \link{http://www.dfo-mpo.gc.ca/csas-sccs/}{www.dfo-mpo.gc.ca/csas-sccs/}\\
\vspace{0.2cm}
ISSN 1919-3815\\
ISBN~\rdISBN{} \hspace{5mm} N\textsuperscript{o} cat.~\rdCatNo{}\\
\vspace{0.2cm}
\copyright{} Sa Majest\'{e} la Roi du chef du Canada,\\
repr\'{e}sent\'{e} par le ministre du minist\`{e}re des P\^{e}ches et des Oc\'{e}ans, \rdWorkDoneYear{}\\
\vspace{0.2cm}
\includegraphics[scale=0.96]{\locRepo/images/recycle.png}
\end{center}

\begin{flushleft}
La pr\'{e}sente publication doit \^{e}tre cit\'{e}e comme suit :
\end{flushleft}

\begin{flushleft}
\citeFr{\rdWorkDoneYear{}/\rdNumber{}}
\end{flushleft}

\begin{flushleft}
\emph{Also available in English:}
\end{flushleft}

\begin{flushleft}
\citeEng{\rdWorkDoneYear{}/\rdNumber{}}
\end{flushleft}

\end{document}
